\newpage
\section{Déroulement du projet}
\label{sec:deroulement}

Dans cette section, nous décrivons comment le projet a été réalisé en équipe : la répartition des tâches, la synchronisation du travail en membres de l'équipe, etc.

\subsection{Réalisation du projet par étapes}
La réalisation du projet s'est déroulée en plusieurs étapes clés, permettant une progression méthodique et organisée. Chaque étape a été soigneusement planifiée pour garantir une avancée cohérente et efficace du projet. Voici les principales étapes suivies :
\begin{enumerate}
\item  Analyse des besoins : Nous avons commencé par une analyse approfondie des besoins du projet, en identifiant les fonctionnalités essentielles et les contraintes à prendre en compte.
\item  Conception initiale : Sur la base des exigences identifiées, nous avons élaboré une première conception du logiciel, définissant l'architecture globale et les grandes lignes de développement.
\item  Implémentation : Une fois la conception validée, nous avons entamé la phase d'implémentation, où chaque membre de l'équipe a contribué à coder les différentes fonctionnalités du logiciel.
\item  Tests et débogage : À mesure que les fonctionnalités étaient développées, nous avons procédé à des tests rigoureux pour détecter et corriger les éventuels bugs ou problèmes de fonctionnement.
\item  Intégration et optimisation : Une fois toutes les fonctionnalités implémentées, nous avons procédé à l'intégration de l'ensemble du système, en veillant à son bon fonctionnement global et à sa performance optimale.
\item  Validation finale : Avant la livraison du produit final, nous avons effectué une validation approfondie pour nous assurer que toutes les exigences du projet étaient satisfaites et que le logiciel répondait aux attentes du client.

\end{enumerate}
\newpage
\subsection{Répartition des tâches entre membres de l'équipe}
\paragraph{}En tant qu'équipe de trois personnes travaillant de manière équitable, nous avons réparti les tâches de manière collaborative et équilibrée. Chaque membre de l'équipe a été chargé de différentes responsabilités tout au long du projet, avec une rotation régulière des rôles pour garantir une participation égale de chacun. Cette approche nous a permis de maximiser l'efficacité de notre travail tout en favorisant un environnement de travail collaboratif et harmonieux.