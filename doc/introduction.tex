\newpage
\section{Introduction}
\label{sec:introduction}

\paragraph{}Ce projet ambitionne de simuler la vision d'un drone volant, dédié à une mission spécifique consistant à détecter et traiter en continu les images du terrain qu'il capture. Conçu pour réagir de manière adaptée dans divers environnements et situations, ce logiciel offre à l'utilisateur une exploration pratique du traitement d'images . Parmi les missions assignées au drone figurent la détection d'incendies, et l'évaluation de l'état des routes, contribuant ainsi à la préservation de l'environnement et à l'amélioration de la sécurité routière.


\subsection{Contexte du projet}
\paragraph{}Le projet "Vision" vise a développer un logiciel capable de simuler la vision d'un drone volant, assigné à des missions spécifiques telles que la détection d'incendies et l'évaluation de l'état des routes. En exploitant le traitement d'images avancé, notre objectif est d'automatiser les actions du drone pour offrir une expérience utilisateur fluide et intuitive. Ce projet s'inscrit dans une démarche visant à utiliser les technologies innovantes pour répondre à des enjeux environnementaux et de sécurité, démontrant ainsi l'impact positif que la technologie peut avoir sur notre société et notre environnement. 

\subsection{Objectif du projet}
\paragraph{}{Notre Objectif est de résoudre des problèmes concrets en utilisant la technologie pour créer un impact positif sur la société et l'environnement. En unissant nos compétences et notre dévouement, nous aspirons à créer un impact durable. Ainsi, nous nous sommes fixés plusieurs objectifs ambitieux : développer un logiciel capable de simuler la vision d'un drone volant et de traiter en continu les images du terrain capturées. Ce logiciel sera doté de capacités avancées de traitement d'images permettant la détection d'incendies et l'évaluation de l'état des routes. En offrant à l'utilisateur une interface conviviale, nous permettrons à chacun d'explorer les aspects pratiques du traitement d'images. 
\subsection{Organisation du rapport}
\paragraph{}Le rapport est structuré de manière à fournir une vue d'ensemble claire et organisée du projet "VISION".Voici un aperçu de la structure du rapport :
\begin{itemize}
    \item \textbf{Introduction :} Cette section présente le contexte du projet, son objectif principal et l'organisation du rapport.
    \item \textbf{Spécification du projet :} Cette partie détaille les notions de base et les contraintes du projet, ainsi que les fonctionnalités attendues du système.
    \item \textbf{Conception et réalisation du projet :} Dans cette section, nous abordons l'architecture globale du logiciel, la conception des classes de données, des traitements et de l'interface graphique.
    \item \textbf{Manuel utilisateur : }Cette partie fournit des instructions détaillées sur l'utilisation du logiciel, notamment comment effectuer différentes tâches et interagir avec l'interface utilisateur.
    \item \textbf{Déroulement du projet :} Ici, nous décrivons comment le projet a été réalisé étape par étape, en détaillant la répartition des tâches entre les membres de l'équipe et en exposant les défis rencontrés et les solutions adoptées.
    \item \textbf{Conclusion et perspectives :} Enfin, cette section résume le travail réalisé, met en évidence les principales conclusions tirées du projet et suggère des pistes d'amélioration pour de futures itérations. 
   
\end{itemize}
Ce rapport est complété par une table des matières qui permet de naviguer facilement entre les différentes sections et sous-sections, offrant ainsi une lecture fluide et structurée du document.












