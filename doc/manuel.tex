\newpage

\section{Manuel utilisateur}
\label{sec:manuel} 

\paragraph{} Le manuel utilisateur constitue une ressource essentielle pour guider les utilisateurs dans l'utilisation efficace et optimale de l'application.Il fournit des informations précieuses pour aider les utilisateurs à tirer la meilleur partie de l'application. Les captures d'écran illustratives sont incluses pour faciliter la compréhension et offrir une expérience utilisateur fluide et intuitive.

\paragraph{Page d'accueil (Figure 3)}La page d'accueil de l'application offre une interface simple et intuitive pour les utilisateurs. Elle présente deux boutons principaux permettant de choisir entre deux situations à traiter : la détection d'incendie et les statistiques routières.


\fig{images/screen1.png}{15cm}{9cm}{Page d'accueil}{Accueil}


\paragraph{Page principale (Figure 4-Détection d'incendie)}Une fois la simulation de détection d'incendie lancée, les utilisateurs ont accès à des statistiques détaillées sur les incendies détectés. Voici ce à quoi les utilisateurs peuvent s'attendre :
\begin{itemize}
    \item Panneau d'informations : Ce panneau affiche les résultats des simulations en cours, fournissant des détails sur les incendies détectés, les mouvements des personnes par rapport au feu, etc.
    \item Boutons d'assistance : Ces boutons permettent d'accéder à l'aide pour comprendre le fonctionnement de l'application et lancer de nouvelles recherches.
    \item Panneau d'image : Ce panneau affiche les images capturées par le drone, mettant en évidence les zones affectées par les incendies et les mouvements des personnes.
    \item Bouton de démarrage : Ce bouton permet de lancer la simulation, affichant de manière récursive les incendies détectés et les mouvements des personnes par rapport au feu.
    \item  Contrôle de la simulation : Les utilisateurs ont la possibilité de mettre en pause et de redémarrer la simulation, ainsi que de régler la vitesse de la simulation à l'aide d'un curseur.
    \item Analyse statistique approfondie :
Les utilisateurs peuvent obtenir des données statistiques détaillées, telles que la répartition des incendies par région, la fréquence des incendies dans le temps, et d'autres métriques pertinentes.
\item Aide et Assistance : 
Pour une meilleure compréhension de l'application et de ses fonctionnalités, les utilisateurs peuvent accéder à l'aide à tout moment en cliquant sur les boutons dédiés. Des instructions détaillées sont fournies pour guider les utilisateurs tout au long de leur expérience.
\item Nouvelle recherche et sortie de l'application :
Les utilisateurs ont la possibilité de lancer de nouvelles recherches pour explorer différentes simulations ou de quitter l'application une fois qu'ils ont terminé leur session. Ces options sont accessibles à partir des boutons prévus à cet effet sur l'interface utilisateur.
\end{itemize}


\fig{images/firedetection.png}{15cm}{9cm}{Page principale-Détection Incedie}{principale}
\newpage
\paragraph{Page principale (Figure 5 - Détection des trous)}Lorsque l'utilisateur choisit de détecter les trous sur la route, une nouvelle fenêtre s'affiche, présentant les éléments suivants :
\begin{itemize}
    \item Image de la route :
Une image de la route est affichée, mettant en évidence les zones potentiellement affectées par des patholes. Cette image permet à l'utilisateur de visualiser visuellement les conditions de la route.
    \item Boutons et fonctionnalités :
Les mêmes boutons et fonctionnalités disponibles pour la détection d'incendie sont présents ici pour assurer une expérience utilisateur cohérente.
Les utilisateurs peuvent mettre en pause et redémarrer la simulation, ainsi que régler la vitesse de la simulation à l'aide d'un curseur.
Un bouton de démarrage est disponible pour lancer la simulation de détection des trous.
    \item Détection des trous :
Pendant la simulation, l'application analyse l'image de la route pour détecter les trous et les zones endommagées.
Les patholes détectées sont mises en évidence sur l'image, permettant à l'utilisateur de les identifier facilement.
\item Panneau d'informations :
Le panneau d'informations affiche en temps réel les résultats de la détection de trous, y compris le nombre total de trous détectées, leur gravité et leur emplacement sur la route.
\item Analyse statistique approfondie :
Les utilisateurs peuvent obtenir des données statistiques détaillées sur l'état de la route, telles que la densité des trous, leur répartition par zone géographique, et d'autres métriques pertinentes.

\fig{images/roadetection.png}{15cm}{9cm}{Page principale-Détection des trous}{principal}
\end{itemize}
